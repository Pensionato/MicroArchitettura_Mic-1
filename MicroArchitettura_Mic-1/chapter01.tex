\chapter{Concetti Preliminari}
\section{Livello di Microarchitettura}
Implementato sopra il livello logico digitale troviamo il livello di microarchitettura.
In specifico tratteremo la microarchittettura di processore Mic-1 realizzata da Andrew S. Tanenbaum.
Il Mic-1 un unità di controllo caratterizzata da varie componenti logiche il cui compito finale è quello di eseguire un microprogramma. Il microprogramma prima di poter essere eseguito verrà interpretato da un ISA (Instruction Set Architecture) ovvero l'architettura dell'insieme di istruzioni, nel nostro caso  tratteremo l'ISA della Java Virtual Machine: l'IJVM. L'IJVM (Integer Java Virtual Machine) come dice il nome è un sottoinsieme di istruzioni della JVM su integer, l'insieme di queste microistuzioni andrà poi a costistuire i microprogrammi. Il Mic-1 per poter consentire la scrittura dell'interprete IJVM ha un suo linguaggio microcodice MAL (Micro-Assembly Language ). Il modello di esecuzione si definisce di fetch-decode-execute, si preleva istruzione per istruzione, si decodifica il codice e si manda in esecuzione.
\subsection{Percorso Dati}
Per percorso dati si intende l'insieme dei collegamenti di input/output alla ALU. I collegamenti son formati dai Bus Dati che sono i canali che permettono la comunicazione e trasporto dell'informazione e rappresentano quelli che fisicamente sono i cavi dei circuiti.Esistono due Bus da 32 linee ciascuno, il Bus B collegato all'output dei registri e all'input della ALU, il Bus C collegato all'output dello shifter e all'input dei registri. Nel nostro percorso dati son presenti dei registri a 32 bit che sono selezionati da due linee di controllo, una per abilitare il bus B e l'altra per abilitare il bus C. I dati in uscita vengono eleborati dalla ALU che genera un output a sua volta eleborato dallo shifter. La ALU contiene al suo interno 32 circuiti combinatori. Ogni circuito combinatorio è capace di eseguire istruzioni logiche di AND, OR e NOT e istruzioni aritmetiche di somma. In entrata alla ALU vi sono 6 linee di controllo. Le prime due determinano l'operazione della ALU, ENA e ENB abilitano i due input (enable A,B), INVA inverte l'input di sinistra e INC crea un riporto (somma 1 al risultato) nel bit meno significativo. La ALU ha due canali in ingresso A e B.
A è collegato al registro H (holding) di mantenimento, B è collegato al Bus B che quindi riceve output da 9 registri. Attraverso lo Shifter il risultato uscente dalla ALU potrebbe non subire nessuna variazione, oppure	potrebbe usare uno Shift Left Logical o uno Shift Right Logical. Lo shift a sinistra (SLL8) trasla il valore a sinistra di un byte e imposta gli 8 bit meno significativi a 0; lo shift a destra (SRA1) trasla il valore di 1 bit a destra e lascia inalterati i bit meno significativi.
\subsection{Ciclo di Clock}
All'inizio di ogni ciclo di clock viene generato un breve impulso. Il fronte di discesa dell'impulso è la parte conosciuta e significativa che gestirà le porte logiche. Il fronte di discesa si divide in una serie di intervalli:
\begin{enumerate}
\item $\Delta$w: Viene selezionato il registro e il suo contenuto viene mandato al bus B
\item $\Delta$x:  Vengono impostati bus B e successivamente il registro H che manterrà i contenuti (in fase di stabilizzazione) prima che vengano inviati alla ALU. 
\item $\Delta$y: ALU e Shifter (che son rimasti sempre attivi) ricevono i dati da elaborare.  
\item $\Delta$z: I risultati passano per il Bus C che li porta ai relativi registri che si aggiorneranno coi nuovi dati/valori.
\end{enumerate}
Al successivo impulso sul fronte iniziale di salita i risultati giungono effettivamente nei registri. Sempre in questo istante il Bus B non riceve più alimentazione dal registro che lo alimentava, per prepararsi al prossimo ciclo. E' importante sottolineare l'importanza di questa rigida e necessaria temporizzazione perchè determina l'intervallo di esecuzione e quindi il costo in tempo. Ai singoli componenti sono estranei i sottocicli di clock, gli stiamo considerando tali perchè rappresentano i limiti in cui i determinati valori possono essere considerati validi. E' compito del progettista assicurarsi che il ciclo di clock avvenga in un tempo sufficientemente esatto,
\subsection{Memoria}
Ci sono diversi modi con cui l'architettura si interfaccia alla memoria: una porta a 32 bit con indirizzi in parole e una porta di 8 bit con indirizzi in byte. I registri MAR (Memory Address Register) e MDR (Memory Data Register) controllano la porta a 32 bit. La porta a 8 bit è controllata dal PC (Program Counter) che legge 1 byte negli 8 bit meno significativi del registro MBR (Memory Buffer Register).
Ciascun registro è controllato da uno o due segnali di controllo, tra questi riconosciamo un segnale di abilitazione output verso il bus B. Il registro MAR non avendo connessione al Bus B e il registro H che è sempre abilitato (essendo l'unica uscita) non hanno questo segnale. L'altro tipo di segnale di controllo per i registri ne determina l'abilitazione a ricevere dati in entrata dal Bus C. Quest'ultimo non è presente nel registro MBR perchè non può esser caricato dal Bus C. Rispettivamente la combinazione di MAR/MDR viene usata per leggere/scrivere dati del livello ISA mentre la combinazione PC/MBR per leggere l'eseguibile a livello ISA. Nell'implementazioni reali esiste un unnica memoria, è necessario quindi consentire a MAR di contare il numero di parole dato che bisogna distinguere dal bus degli indirizzi che specificano byte. I bit di MAR non vengono direttamente mappati sulle 32 linee (da 0 a 31). Si esegue un ''shift'' per cui ilbit 0 di MAR si collega alla linea 2 del bus indirizzi e i due bit più alti vengono scartati in quanto non necessariamente significativi. I dati letti dalla porta ad 8 bit vengono restituiti a MBR (registro a 8 bit), a sua volta MBR viene copiato nel Bus B in due modi: con o senza segno (determinato da due segnali di controllo a seconda del metodo). Nella parola a 32 bit quando viene richiesto un valore senza segno negli 8 bit meno significativi si memorizzerà il valore di MBR e i restanti 24 verranno posti a 0. Per convertire il registro MBR a 8 in 32 bit si può anche trattarlo come se fosse un valore compreso tra -128 e +127 e convertire quindi questo ultimo valore in una parola da 32 bit. Questa operazione si chiama estensione del segno e si esegue  duplicando i bit del segno di MBR nei 24 bit più alti del Bus B. Se il bit più a sinistra di MBR vale 0 i 24 bit più alti saranno tutti a 0, se vale 1 verrano posti a 1.
\section{MAL (Micro Assembly Language)}
Possiamo distinguere 29 segnali divisibili in 5 gruppi per scopo.
\begin{itemize}
\item 9 segnali di controllo scrittura del Bus C sui registri
\item 9 segnali di abilitazione dei registri sul Bus B
\item 8 segnali per funzioni ALU e Shifter
\item 2 segnali per indicare alla memoria di leggere/scrivere tramite MAR/MDR
\item 1 segnale per indicare il prelievo della memoria tramite PC o MBR
\end{itemize}
In un ciclo di percorso dati i dati passano dai registri al bus B che li porta sulla ALU che a sua volta rende il risultato ai registri attraverso il Bus C. Un'eventuale lettura dati (su una delle due porte) trasmette dati che non possono essere usati al ciclo successivo ma al ciclo successivo+1. Questo perchè i dati vengono caricati nel MAR a fine ciclo quindi non possiamo aspettarci che siano atomicamente disponibili in MDR all'inizio del ciclo successivo. L'operazione di lettura richiede un ciclo e si possono eseeguire letture in sequenza durante cicli consecutivi. Si possono usare allo stesso tempo le porte di memoria ma leggere e scrivere simultaneamente genera risultati indefiniti. Mentre sul Bus B è abilitato sempre e solo un registro, su quello C ne possiamo abilitare diversi. L'informazione del Bus B viene codificatain 4 bit e tramite un decodificatore (4 a 16) si generano 16 segnali di controllo (7 inutilizzati).
Il percorso dati in definitiva si controlla con 9+4+8+2+1=24 segnali(bit). In aggiunta a questi 24 bit di controllo ciclo sono necessari anche i campi NEXTADDRESS e JAM per capire cosa dev'essere fatto nel ciclo successivo. Per determinare i collegamenti e incroci tra chip (per ottimizzarli) si è sviluppato un possibile formato di 6 gruppi di un totale di 36 segnali.
\begin{itemize}
\item Addr - Contiene indirizzo (address) di una possibile successiva Microistruzione
\item JAM - Determina la selezione della successiva Microistruzione
\item ALU - Seleziona le funzioni di ALU e Shifter
\item C - Seleziona i registri sui quali il Bus C deve scrivere
\item Mem - Seleziona la funzione della Memoria
\item B - Seleziona codifica sorgente Bus B (0=MDR, 1=PC, 2=MBR, 3=MBRU, 4=SP, 5=LV, 6=CPP, 7=TOS, 8=OPC, 9-15 none)
\end{itemize}
\section{Unità di controllo microprogrammata}
Nella descrizione della nostra architettura molte volte useremo circuiti standard a scopi didattici non ottimizzati. 
Introduciamo ora il sequenzializzatore che determina la sequenza di operazioni per eseguire singolarmente le istruzioni ISA. Il sequenzializzatore ad ogni ciclo ci dà lo stato di ogni segnale di controllo del sistema e l'indirizzo della prossima microistruzione da eseguire. Il Mic-1 si può suddividere in due grandi parti: il percorso dati che è stato visto e la sezione di controllo. Il nucleo della sezione di controllo è la memoria di controllo. La memoria di controllo è un circuito (una serie di porte logiche) che memorizza Microistruzioni (da non confondere con istruzioni ISA). Nel Mic-1 la memoria è grande 512 parole (consistono nei 36 bit della Microistruzione).
Le sequenze dei microprogrammi tendono ad essere brevi e per avere una flessibilità maggiore ogni microistruzione (solitamente) specifica il suo successore.
La memoria di controllo ha: un registro degli indirizzi MPC (MicroProgram Counter) ,il cui nome può risultare ambiguo perchè le istruzioni in realtà non hanno un ordine preciso, e un registro dei dati detto MIR (MicroIstruction Register). Il MIR contiene la microistruzione in esecuzione corrente, la quale determinerà il percorso dati.
Il percorso totale del Mic-1 si può così descrivere:
\begin{itemize}
\item Ad inizio ciclo di clock (fronte di discesa) la parola nella memoria di controllo puntata da MPC viene trasferita in MIR ($\Delta$w tempo)
\item Negli istanti dopo i segnali si propagano nel bus dati (registri > Bus B > ALU > Bus C) fino ad arrivare al secondo sottociclo
\item Dopo un secondo $\Delta$y successivo tutto si stabilizza nel circuito e i valori di N e Z (uscite ALU) vengono salvati in una coppia di flip-flop ad 1 bit.
\item I bit dei flip-flop vengono salvati in un tempo di fine ciclo (fronte salita clock)
\item In un 3 sottociclo si svolgeranno le attività della ALU e dello Shifter (l'output della ALU non si memorizza ma finisce nello Shifter)
\item Dopo un ulteriore $\Delta$z dallo Shifter si va al Bus C e l'output raggiunge i registri
\item Nel quarto sottociclo vengono caricati i registri e i flip-flop N e Z, i risultati delle operazioni di memoria precedenti son quindi disponibili e lo stato di MPC viene aggiornato.
\end{itemize}















